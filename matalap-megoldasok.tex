\documentclass[12pt,a4paper,fleqn]{article}
\usepackage[a4paper,left=3cm,right=2cm,top=2.5cm,bottom=2.5cm]{geometry}
\usepackage{amsfonts}
\usepackage{mathtools}
\usepackage{fancyhdr}
\usepackage{enumerate}
\setlength{\mathindent}{0pt}

\author{}
\pagestyle{fancy}
\fancyhf{}
\lhead{\leftmark}
\rhead{\thepage}

\title{Matematikai Alapok \\ \textbf{Megoldások}}

\begin{document}
\maketitle
\newpage

\section{Algebrai és Gyökös kifejezések I.}
\setcounter{subsection}{1}
\subsection{Feladatok}
\subsubsection{Órai feladatok}
\textbf{3. (b)}
\begin{flalign*}
  & \frac{a}{a^3+a^2b+ab^2+b^3} + \frac{b}{a^3-a^2b+ab^2-b^3} + \frac{1}{a^2-b^2} - \frac{1}{a^2+b^2} - \frac{a^2+3b^2}{a^4-b^4} = 0  \\\\
  & \frac{a}{a^2(a+b)+b^2(a+b)} + \frac{b}{a^2(a-b)+b^2(a-b)} + \frac{1}{(a+b)(a-b)} - \frac{1}{a^2+b^2} - \frac{a^2+3b^2}{(a^2-b^2)(a^2+b^2)} = 0 \\\\
  & \frac{a}{(a+b)(a^2+b^2)} + \frac{b}{(a-b)(a^2+b^2)} + \frac{1}{(a+b)(a-b)} - \frac{1}{a^2+b^2} - \frac{a^2+3b^2}{(a+b)(a-b)(a^2+b^2)} = 0 \\\\
  & \frac{a(a-b)+b(a+b)+(a^2+b^2)-(a+b)(a-b)-(a^2+3b^2)}{(a+b)(a-b)(a^2+b^2)} = 0 \\\\
  & \frac{a^2-ab+ab+b^2+a^2+b^2-a^2+b^2-a^2-3b^2}{(a+b)(a-b)(a^2+b^2)} = 0
\end{flalign*} \\
\textbf{4. (a)}
\begin{enumerate}
  \item Első lehetőség: $a+b+c=0$ kiemelhető így $(a+b+c) \cdot (a^2+b^2-ab)$
  \item Második lehetőség: $(a+b+c)$-ből kifejezzük az egyiket $\rightarrow c = -a - b$ és ezt kell behelyettesíteni. \\
  $a^3 + a^2(-a-b) -ab(-a-b) + b^2(-a-b) + b^3$ \\
  $ = a^3 - a^3 - a^2b + a^2b +ab^2 - ab^2 - b^3 + b^3$ \\
  $ = 0$ \\
\end{enumerate}
\textbf{6.} \\\\
$$ \dfrac{x^3-x-y^3+y+xy^2-x^2y}{x^3+x-y^3-y+xy^2-x^2y} =
  \dfrac{xy(y-x)+(y-x)+(x-y)(x^2+xy+y^2)}{xy(y-x)-(y-x)+(x-y)(x^2+xy+y^2)} $$
  $$= \dfrac{(y-x)(xy+1-x^2-y^2-xy)}{(y-x)(xy-1-x^2-y^2-xy)}
  = \dfrac{1-x^2-y^2}{-1-x^2-y^2} = \dfrac{-1(-1+x^2+y^2)}{-1(1+x^2+y^2)} = \dfrac{x^2+y^2-1}{x^2+y^2+1}$$
Ha $x = \dfrac{k(1-z^2)}{(1+z^2)}$ akkor $y = \dfrac{2k \cdot z}{1+z^2}$.
Bizonyítsuk be hogy behelyettesítés után a kifejezés nem függ a $z$ értékétől (vagyis ki fog esni)!
$$ \dfrac{x^2+y^2-1}{x^2+y^2+1} = \dfrac{x^2+y^2+1-2}{x^2+y^2+1} = 1 - \dfrac{2}{x^2+y2+1} $$
Behelyettesítve:
$$ x^2+y=\dfrac{(k(1-z^2))^2}{(1+z^2)^2} + \dfrac{(2kz)^2}{(1+z^2)^2} = \dfrac{k^2(1-z^2)^2+4k^2z^2}{(1+z^2)^2}$$
$$ = \dfrac{k^2(1+z^4-2z^2+4z^2)}{(1+z^2)^2} = \dfrac{k^2(z^2+1)^2}{(1+z^2)^2} = k^2$$
Így $1-\dfrac{2}{k^2+1}$ ami valóban független a $z$-től. \\\\
\textbf{12. (c)} \\\\
$$ \left(\sqrt{x} - \dfrac{\sqrt{xy}+y}{\sqrt{x}+\sqrt{y}}\right) \cdot \left( \dfrac{\sqrt{x}}{\sqrt{x}+\sqrt{y}} + \dfrac{\sqrt{y}}{\sqrt{x}-\sqrt{y}} + \dfrac{2\sqrt{xy}}{x-y} \right) $$
$$ = \left(\dfrac{\sqrt{x}(\sqrt{x}+\sqrt{y})-(\sqrt{xy}+y)}{\sqrt{x} + \sqrt{y}} \right) \cdot \left( \dfrac{\sqrt{x}(\sqrt{x}-\sqrt{y})+\sqrt{y}(\sqrt{x}+\sqrt{y})+2\sqrt{xy}}{(\sqrt{x}-\sqrt{y})(\sqrt{x}+\sqrt{y})} \right)$$
$$ = \left( \dfrac{x-y}{\sqrt{x}+\sqrt{y}} \right) \cdot \left( \dfrac{x-\sqrt{xy}+\sqrt{xy}+y+2\sqrt{xy}}{(\sqrt{x}-\sqrt{y})(\sqrt{x}+\sqrt{y})} \right)$$
$$ = \left( \dfrac{x-y}{\sqrt{x}+\sqrt{x}} \cdot \left( \dfrac{x+y+2\sqrt{xy}}{(\sqrt{x}-\sqrt{y})(\sqrt{x}+\sqrt{y})} \right) \right) = \left( \dfrac{x-y}{\sqrt{x}+\sqrt{y}} \cdot \left( \dfrac{(\sqrt{x}+\sqrt{y})^2}{(\sqrt{x}+\sqrt{y})(\sqrt{x}-\sqrt{y})} \right) \right)$$
$$ = \dfrac{x-y}{\sqrt{x} - \sqrt{y}} = \sqrt{x} + \sqrt{y}$$ (mert: $x-y=(\sqrt{x}-\sqrt{y}(\sqrt{x}+\sqrt{x}))$) \\\\
\textbf{19. (a)} \\\\
$$ x_0 = 2 \quad P(x)=3x^2 - 7x + 2 $$
Ha egy polinomnak $c$ gyöke akkor $f(x)=(x-c) \cdot g(x)$ \\
$x^2-3x+2=0$-nak $x_1=1$ és $x_2=2$ gyökök. \\
$x^2-3x+3=(x-1)(x-2)$ \\
$3x^2-7x+2 \quad x_0=2$ \\
$ 3x^2-7x+2=(x-2)(3x-1) $ \\\\
\textbf{19. (b)} \\\\
$ x_0 \quad 2x^3-4x^2-18 = (x-3)(...)$ \\
$ (x-3)(ax^2+bx+c) = ax^3+bx^2+cx-3ax^2-3vx-3c = ax^3+(b-3a)x^2+(c-3b)x-3c $ \\
Így $a=2$, $b=7$, $b-3a=4$ vagy $c-3b=0 \Rightarrow b=2 \Longrightarrow (2x^2+2x+6) \Rightarrow (x-3)(2x^2+6x+2)$ \\\\
\textbf{19. (c)} \\\\
$2x^4-5x^3-6x^2+3x+2 \quad x_0=-1 \Longrightarrow (x+1)(2x^3-7x^2+x+2)$

\section{Másodfokú egyenletek, egyenlőtlenségek}
\setcounter{subsection}{1}
\subsection{Feladatok}
\subsubsection{Órai feladatok}
\textbf{3. (b)} \\\\
$$ \dfrac{3x^2+7x-4}{x^2+2x-3} < 2 $$
\begin{enumerate}[i.]
  \item ha $x^2+2x-3 > 0$ \\
  $x^2+2x-3 = (x+3)(x-1)$ ha $x>1$ vagy $x<-3$ \\
  $3x^2+7x-4 < 2x^2+4x-6$ \\
  $x^2 + 3x + 2 < 0 \Rightarrow x_1=-1$ és $x_2=-2$. $x \in (-2, -1)$ ez nem megoldás mert a kettnek nincs közös része.
  \item ha $x^2+2x-3 < 0$: $x \in (-3, 1)$ \\
  $x^2+3x+2 > 0$: $x<-2$ vagy $x>-1$ \\
  $x \in (-3, -2)$ vagy $x \in (-1, 1)$
\end{enumerate}
\textbf{4. (c)} \\\\
$$ (p^2-1)x^2 + 2(p-1)x + 1 > 0 $$
Ha $p=1$ akkor teljesűl.
Ha a függvény grafikonja metszi az $x$ tengelyt van gyöke. Ha két helyen metszi két gyöke van.
Ha a diszkrimináns kisebb mint 0 nincs megoldása. Ha pozitív akkor két megoldása van. Ha nulla akkor egy az $\mathbb{R}$-ben \\
$b^2-4ac=(2(p-1))^2-3 \cdot 1 \cdot (p^2)-1 < 0$ \\
$ 4(p-1)^2 - 4(p^2 - 1) < 0 $ \\
$ 4(p^2 - 2p + 1) - 4p^2 + 4 < 0 $ \\
$ 4p^2 - 8p + 4 - 4p^2 + 4 < 0 $ \\
$ -8p + 8 < 0 $ \\
$ 1 < p $ Ekkor negatív a diszkrimináns tehát nem lesz megoldás.
\section{Algebrai és gyökös kifejezések II.}
\setcounter{subsection}{1}
\subsection{Feladatok}
\subsubsection{Órai feladatok}
\textbf{1. (c)} \\\\
$$ \dfrac{2x^2-13x-7}{8x^3+1} = \dfrac{(x-7)(2x+1)}{(2x+1)(4x^2-2x+1)} = \dfrac{(x-7)}{(x^2-2x+1)}$$
$2x^2-13x-7=0 \rightarrow x_1 = 7$, $x_2 = -\dfrac{1}{2}$ \\
$2x^2-13x-7=2(x-7)(x+\dfrac{1}{2})=(x-7)(2x+1)$ \\
$ 8x^3 + 1 = (2x)^3 + (1)^3 = (2x+1)(4x^2-2x+1)$ \\\\
\textbf{1. (e)} \\\\
\begin{align*}
  \begin{split}
    & \dfrac{2}{x^2-1} -\dfrac{3}{x^3-1} = \dfrac{2}{(x-1)(x+1)} - \dfrac{3}{(x-1)(x^2+x+1)} = \dfrac{2(x^2+x+1)-3(x+1)}{(x-1)(x+1)(x^2+x+1)} \\[10pt]
    &= \dfrac{2x^2+2x+2-3x-3}{(x-1)(x+1)(x^2+x+1)} = \dfrac{2x^2-x-1}{(x-1)(x+1)(x^2+x+1)} \\[10pt]
    &= \dfrac{2(x-1)(x+\dfrac{1}{2})}{(x-1)(x+1)(x^2+x+1)} = \dfrac{2x+1}{(x+1)(x^2+x+1)}
  \end{split}
\end{align*}
\textbf{3. (b)} \\\\
$|2x-7|-|2x+7|=x+15 \quad x \in \mathbb{R}$
\begin{enumerate}
  \item eset: mindkettő negatív
  \begin{align*}
    \begin{split}
      &2x - 7 < 0 \Rightarrow x < \dfrac{7}{2} \\
      &2x + 7 < 0 \Rightarrow x < -\dfrac{7}{2} \\
      &\Longrightarrow x \in (-\infty, \dfrac{7}{2})
    \end{split}
  \end{align*}
  \item eset
  \begin{align*}
    \begin{split}
      &2x + 7 \geq 0 \Rightarrow x \geq -\dfrac{7}{2} \\
      &2x - 7 \geq 0 \Rightarrow x \geq \dfrac{7}{2} \\
      &\Longrightarrow x \in [\dfrac{7}{2}, \infty)
    \end{split}
  \end{align*}
  \item eset ($2x+7$ biztos hogy nagyobb mint $2x-7$) \\
  $2x-7<0$ és $2x+7 \geq 0 \Rightarrow x \in \left[ -\dfrac{7}{2}, \dfrac{7}{2} \right)$
\end{enumerate}
Kiszámoljuk
\begin{enumerate}
  \item esetet: ($-1$-szeresét kell venni mert negatív) \\
  \begin{align*}
    \begin{split}
      -(2x-7)+(-(2x+7)) &= x + 15 \\
      7-2x-2x-7&=x+15 \\
      -4x&=x+15 \\
      x &=-3
    \end{split}
  \end{align*}
  Ez nem megoldás mert $-3 \notin (-\infty, -\dfrac{7}{2})$.
  \item esetet: (mindkettő pozitív ezért nem változik az előjel)
  \begin{align*}
    \begin{split}
      2x-7+2x+7 &= x + 15 \\
      x &= 5
    \end{split}
  \end{align*}
  Ez megoldás mert $5 \in (\dfrac{7}{2}, \infty)$.
  \item esetet:
  \begin{align*}
    \begin{split}
      -(2x+7)+2x+7 &= x + 15 \\
      -2x-7+2x+7 &= x + 15 \\
      x &= -1
    \end{split}
  \end{align*}
  Ez megoldás mert $5 \in \left[ -\dfrac{7}{2}, \dfrac{7}{2} \right)$.
\end{enumerate}
Tehát két megoldás van: $5$ és $-1$.



\end{document}
