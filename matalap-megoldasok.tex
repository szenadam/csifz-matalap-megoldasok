\documentclass[12pt,a4paper,fleqn]{article}
\usepackage[a4paper,left=3cm,right=2cm,top=2.5cm,bottom=2.5cm]{geometry}
\usepackage{amsfonts}
\usepackage{mathtools}
\setlength{\mathindent}{0pt}

\author{Et Alii}
\pagestyle{headings}

\title{Matemematikai Alapok \\ \textbf{Megoldások}}

\begin{document}
\maketitle
\newpage

\section{Algebrai és Gyökös kifejezések I.}
\subsection{Feladatok}
\subsubsection{Órai feladatok}
3. (b)

\begin{flalign*}
  & \frac{a}{a^3+a^2b+ab^2+b^3} + \frac{b}{a^3-a^2b+ab^2-b^3} + \frac{1}{a^2-b^2} - \frac{1}{a^2+b^2} - \frac{a^2+3b^2}{a^4-b^4} = 0  \\\\
  & \frac{a}{a^2(a+b)+b^2(a+b)} + \frac{b}{a^2(a-b)+b^2(a-b)} + \frac{1}{(a+b)(a-b)} - \frac{1}{a^2+b^2} - \frac{a^2+3b^2}{(a^2-b^2)(a^2+b^2)} = 0 \\\\
  & \frac{a}{(a+b)(a^2+b^2)} + \frac{b}{(a-b)(a^2+b^2)} + \frac{1}{(a+b)(a-b)} - \frac{1}{a^2+b^2} - \frac{a^2+3b^2}{(a+b)(a-b)(a^2+b^2)} = 0 \\\\
  & \frac{a(a-b)+b(a+b)+(a^2+b^2)-(a+b)(a-b)-(a^2+3b^2)}{(a+b)(a-b)(a^2+b^2)} = 0 \\\\
  & \frac{a^2-ab+ab+b^2+a^2+b^2-a^2+b^2-a^2-3b^2}{(a+b)(a-b)(a^2+b^2)} = 0
\end{flalign*}

\end{document}