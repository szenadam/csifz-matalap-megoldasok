\documentclass[12pt,a4paper,fleqn]{article}
\usepackage[a4paper,left=3cm,right=2cm,top=2.5cm,bottom=2.5cm]{geometry}
\usepackage{tikz}
\usepackage{pgfplots}
\pgfplotsset{compat=newest}
\usepackage{wrapfig}
\usepackage{amsfonts}
\usepackage{mathtools}
\usepackage{nicefrac}
\usepackage{fancyhdr}
\usepackage{enumerate}
\usepackage[utf8x]{inputenc}
\usepackage[hungarian]{babel}
\usepackage[hidelinks]{hyperref}

% Indent/spacing settings
\setlength{\mathindent}{0pt}
\setlength\parindent{0pt}
\renewcommand{\baselinestretch}{1.3}

% Header settings
\pagestyle{fancy}
\fancyhf{}
\lhead{\leftmark}
\rhead{\thepage}

% Title/Author
\author{}
\title{Matematikai Alapok \\ \textbf{Megoldások}}

% Custom commands
\newcommand{\myparagraph}[1]{\paragraph{#1}\mbox{}}

\setcounter{tocdepth}{4}

\begin{document}
\maketitle
\thispagestyle{empty}
\clearpage

\section{Algebrai és Gyökös kifejezések I.}
\setcounter{subsection}{1}
\subsection{Feladatok}
\subsubsection{Órai feladatok}

\myparagraph{3. (b)}
\begin{flalign*}
  & \frac{a}{a^3+a^2b+ab^2+b^3} + \frac{b}{a^3-a^2b+ab^2-b^3} + \frac{1}{a^2-b^2} -
  \frac{1}{a^2+b^2} - \frac{a^2+3b^2}{a^4-b^4} = 0  \\
  & \frac{a}{a^2(a+b)+b^2(a+b)} + \frac{b}{a^2(a-b)+b^2(a-b)} + \frac{1}{(a+b)(a-b)} -
  \frac{1}{a^2+b^2} - \frac{a^2+3b^2}{(a^2-b^2)(a^2+b^2)} = 0 \\
  & \frac{a}{(a+b)(a^2+b^2)} + \frac{b}{(a-b)(a^2+b^2)} + \frac{1}{(a+b)(a-b)} -
    \frac{1}{a^2+b^2} - \frac{a^2+3b^2}{(a+b)(a-b)(a^2+b^2)} = 0 \\
  & \frac{a(a-b)+b(a+b)+(a^2+b^2)-(a+b)(a-b)-(a^2+3b^2)}{(a+b)(a-b)(a^2+b^2)} = 0 \\\\
  & \frac{a^2-ab+ab+b^2+a^2+b^2-a^2+b^2-a^2-3b^2}{(a+b)(a-b)(a^2+b^2)} = 0
\end{flalign*}

\myparagraph{4. (a)}
\begin{enumerate}
  \item Első lehetőség: $a+b+c=0$ kiemelhető így $(a+b+c) \cdot (a^2+b^2-ab)$
  \item Második lehetőség: $(a+b+c)$-ből kifejezzük az egyiket
    $\rightarrow c = -a - b$ és ezt kell behelyettesíteni.
  \[ a^3 + a^2(-a-b) -ab(-a-b) + b^2(-a-b) + b^3 = \]
  \[ a^3 - a^3 - a^2b + a^2b +ab^2 - ab^2 - b^3 + b^3 = 0 \]
\end{enumerate}

\myparagraph{6.}
\[ \dfrac{x^3-x-y^3+y+xy^2-x^2y}{x^3+x-y^3-y+xy^2-x^2y} =
  \dfrac{xy(y-x)+(y-x)+(x-y)(x^2+xy+y^2)}{xy(y-x)-(y-x)+(x-y)(x^2+xy+y^2)} = \]
\[
  \dfrac{(y-x)(xy+1-x^2-y^2-xy)}{(y-x)(xy-1-x^2-y^2-xy)} =
  \dfrac{1-x^2-y^2}{-1-x^2-y^2} = \dfrac{-1(-1+x^2+y^2)}{-1(1+x^2+y^2)} =
  \dfrac{x^2+y^2-1}{x^2+y^2+1}
\]

Ha $x = \dfrac{k(1-z^2)}{(1+z^2)}$ akkor $y = \dfrac{2k \cdot z}{1+z^2}$
Bizonyítsuk be hogy behelyettesítés után a kifejezés nem függ a $z$ értékétől
(vagyis ki fog esni)!

\[ \dfrac{x^2+y^2-1}{x^2+y^2+1} = \dfrac{x^2+y^2+1-2}{x^2+y^2+1} =
  1 - \dfrac{2}{x^2+y2+1} \]

Behelyettesítve:
\[ x^2+y=\dfrac{(k(1-z^2))^2}{(1+z^2)^2} + \dfrac{(2kz)^2}{(1+z^2)^2} =
  \dfrac{k^2(1-z^2)^2+4k^2z^2}{(1+z^2)^2} \]
\[ = \dfrac{k^2(1+z^4-2z^2+4z^2)}{(1+z^2)^2} = \dfrac{k^2(z^2+1)^2}{(1+z^2)^2} = k^2 \]

Így $1-\dfrac{2}{k^2+1}$ ami valóban független a $z$-től.


\myparagraph{12. (c)}
\[ \left(\sqrt{x} - \dfrac{\sqrt{xy}+y}{\sqrt{x}+\sqrt{y}}\right) \cdot
  \left( \dfrac{\sqrt{x}}{\sqrt{x}+\sqrt{y}} + \dfrac{\sqrt{y}}{\sqrt{x}-\sqrt{y}} +
  \dfrac{2\sqrt{xy}}{x-y} \right) = \]
\[
  \left(\dfrac{\sqrt{x}(\sqrt{x}+\sqrt{y})-(\sqrt{xy}+y)}{\sqrt{x} + \sqrt{y}} \right) \cdot
  \left( \dfrac{\sqrt{x}(\sqrt{x}-\sqrt{y})+\sqrt{y}(\sqrt{x}+\sqrt{y})+
  2\sqrt{xy}}{(\sqrt{x}-\sqrt{y})(\sqrt{x}+\sqrt{y})} \right) =
\]
\[ \left( \dfrac{x-y}{\sqrt{x}+\sqrt{y}} \right) \cdot \left(
  \dfrac{x-\sqrt{xy}+\sqrt{xy}+y+2\sqrt{xy}}{(\sqrt{x}-\sqrt{y})(\sqrt{x}+\sqrt{y})} \right) =
  \dfrac{x-y}{\sqrt{x}+\sqrt{x}} \cdot \left( \dfrac{x+y+2\sqrt{xy}}{(\sqrt{x}-
  \sqrt{y})(\sqrt{x}+\sqrt{y})} \right) \]
\[ = \dfrac{x-y}{\sqrt{x}+\sqrt{y}} \cdot \left(
  \dfrac{(\sqrt{x}+\sqrt{y})^2}{(\sqrt{x}+\sqrt{y})(\sqrt{x}-\sqrt{y})} \right) =
  \dfrac{x-y}{\sqrt{x} - \sqrt{y}} = \sqrt{x} + \sqrt{y} \]
( mert: $x-y=(\sqrt{x}-\sqrt{y})(\sqrt{x}+\sqrt{y})$ )


\myparagraph{19. (a)}
\[ x_0 = 2 \quad P(x)=3x^2 - 7x + 2 \]
\[ \text{Ha egy polinomnak } c \text{ gyöke akkor } f(x)=(x-c) \cdot g(x) \]
\[ x^2-3x+2=0\text{-nak } x_1=1 \text{ és } x_2=2 \text{ gyökök.} \]
\[ x^2-3x+3=(x-1)(x-2) \]
\[ 3x^2-7x+2 \quad x_0=2 \]
\[ 3x^2-7x+2=(x-2)(3x-1) \]


\myparagraph{19. (b)}
\[ x_0 \quad 2x^3-4x^2-18 = (x-3)(...) \]
\[ (x-3)(ax^2+bx+c) = ax^3+bx^2+cx-3ax^2-3vx-3c = ax^3+(b-3a)x^2+(c-3b)x-3c \]
Így $a=2$, $b=7$ és $b-3a=4$ vagy $c-3b=0 \Rightarrow b=2 \Longrightarrow
  (2x^2+2x+6) \Rightarrow (x-3)(2x^2+6x+2)$


\myparagraph{19. (c)}
\[ 2x^4-5x^3-6x^2+3x+2 \quad x_0=-1 \Longrightarrow (x+1)(2x^3-7x^2+x+2) \]



\section{Másodfokú egyenletek, egyenlőtlenségek}
\setcounter{subsection}{1}
\subsection{Feladatok}
\subsubsection{Órai feladatok}


\myparagraph{3. (b)}
\[ \dfrac{3x^2+7x-4}{x^2+2x-3} < 2 \]
\begin{enumerate}[i.]
  \item ha $x^2+2x-3 > 0$
  \[ x^2+2x-3 = (x+3)(x-1)$ ha $x>1$ vagy $x<-3 \]
  \[ 3x^2+7x-4 < 2x^2+4x-6 \]
  $x^2 + 3x + 2 < 0 \Rightarrow x_1=-1$ és $x_2=-2$. $x \in (-2, -1)$ ez nem megoldás
  mert a kettőnek nincs közös része.
  \item ha $x^2+2x-3 < 0$: $x \in (-3, 1)$ \\
  $x^2+3x+2 > 0$: $x<-2$ vagy $x>-1$ \\
  $x \in (-3, -2)$ vagy $x \in (-1, 1)$
\end{enumerate}


\myparagraph{4. (c)}
\[ (p^2-1)x^2 + 2(p-1)x + 1 > 0 \]
Ha $p=1$ akkor teljesül.
Ha a függvény grafikonja metszi az $x$ tengelyt van gyöke. Ha két helyen metszi két gyöke van.
Ha a diszkrimináns kisebb mint 0 nincs megoldása. Ha pozitív akkor két megoldása van.
Ha nulla akkor egy az $\mathbb{R}$-ben.
\[ b^2-4ac=(2(p-1))^2-3 \cdot 1 \cdot (p^2)-1 < 0 \]
\[ 4(p-1)^2 - 4(p^2 - 1) < 0 \]
\[ 4(p^2 - 2p + 1) - 4p^2 + 4 < 0 \]
\[ 4p^2 - 8p + 4 - 4p^2 + 4 < 0 \]
\[ -8p + 8 < 0 \]
$ 1 < p $ Ekkor negatív a diszkrimináns tehát nem lesz megoldás.



\section{Algebrai és gyökös kifejezések II.}
\setcounter{subsection}{1}
\subsection{Feladatok}
\subsubsection{Órai feladatok}


\myparagraph{1. (c)}
\[ \dfrac{2x^2-13x-7}{8x^3+1} = \dfrac{(x-7)(2x+1)}{(2x+1)(4x^2-2x+1)} = \dfrac{(x-7)}{(x^2-2x+1)} \]
\[ 2x^2-13x-7=0 \rightarrow x_1 = 7$, $x_2 = -\dfrac{1}{2} \]
\[ 2x^2-13x-7=2(x-7)(x+\dfrac{1}{2})=(x-7)(2x+1) \]
\[ 8x^3 + 1 = (2x)^3 + (1)^3 = (2x+1)(4x^2-2x+1) \]


\myparagraph{1. (e)}
\begin{align*}
  \begin{split}
    & \dfrac{2}{x^2-1} -\dfrac{3}{x^3-1} = \dfrac{2}{(x-1)(x+1)} - \dfrac{3}{(x-1)(x^2+x+1)} =
        \dfrac{2(x^2+x+1)-3(x+1)}{(x-1)(x+1)(x^2+x+1)} \\[10pt]
    &= \dfrac{2x^2+2x+2-3x-3}{(x-1)(x+1)(x^2+x+1)} = \dfrac{2x^2-x-1}{(x-1)(x+1)(x^2+x+1)} \\[10pt]
    &= \dfrac{2(x-1)(x+\dfrac{1}{2})}{(x-1)(x+1)(x^2+x+1)} = \dfrac{2x+1}{(x+1)(x^2+x+1)}
  \end{split}
\end{align*}


\myparagraph{3. (b)}
\[ |2x-7|-|2x+7|=x+15 \quad x \in \mathbb{R} \]
\begin{enumerate}
  \item eset: mindkettő negatív
  \begin{align*}
    \begin{split}
      &2x - 7 < 0 \Rightarrow x < \dfrac{7}{2} \\
      &2x + 7 < 0 \Rightarrow x < -\dfrac{7}{2} \\
      &\Longrightarrow x \in (-\infty, \dfrac{7}{2})
    \end{split}
  \end{align*}
  \item eset: mindkettő pozitív
  \begin{align*}
    \begin{split}
      &2x + 7 \geq 0 \Rightarrow x \geq -\dfrac{7}{2} \\
      &2x - 7 \geq 0 \Rightarrow x \geq \dfrac{7}{2} \\
      &\Longrightarrow x \in [\dfrac{7}{2}, \infty)
    \end{split}
  \end{align*}
  \item eset ($2x+7$ biztos hogy nagyobb mint $2x-7$)
  \[ 2x-7<0$ és $2x+7 \geq 0 \Rightarrow x \in \left[ -\dfrac{7}{2}, \dfrac{7}{2} \right) \]
\end{enumerate}
Kiszámoljuk
\begin{enumerate}
  \item esetet: ($-1$-szeresét kell venni mert negatív)
  \begin{align*}
    \begin{split}
      -(2x-7)+(-(2x+7)) &= x + 15 \\
      7-2x-2x-7&=x+15 \\
      -4x&=x+15 \\
      x &=-3
    \end{split}
  \end{align*}
  Ez nem megoldás mert $-3 \notin (-\infty, -\dfrac{7}{2})$.
  \item esetet: (mindkettő pozitív ezért nem változik az előjel)
  \begin{align*}
    \begin{split}
      2x-7+2x+7 &= x + 15 \\
      x &= 5
    \end{split}
  \end{align*}
  Ez megoldás mert $5 \in (\dfrac{7}{2}, \infty)$.
  \item esetet:
  \begin{align*}
    \begin{split}
      -(2x+7)+2x+7 &= x + 15 \\
      -2x-7+2x+7 &= x + 15 \\
      x &= -1
    \end{split}
  \end{align*}
  Ez megoldás mert $-1 \in \left[ -\dfrac{7}{2}, \dfrac{7}{2} \right)$.
\end{enumerate}
Tehát két megoldás van: $5$ és $-1$.


\myparagraph{3. (g)}
\[ |2x-1| < |x-1| \]
Itt érdemes grafikont rajzolni és abból hamar rá lehet jönni hogy milyen esetek vannak.
\[ 2x-1 < 1-x \]
\[ x < \dfrac{2}{3} \]


\myparagraph{6. (a)}
\[ \sqrt{x+1} - \sqrt{9-x} = \sqrt{2x-12} \]
Először kikötést kell tenni!
\[
\begin{rcases*}
  x+1 \geq 0 \rightarrow x \geq -1 \\
  9-x \geq 0 \rightarrow 9 \geq x \\
  2x-12 \geq 0 \rightarrow x \geq 6
\end{rcases*} \text{Tehát } x \in [6, 9]
\]

Ezután lehet négyzetre emelni.
\[ x+1+9-x-2(\sqrt{x+1} \cdot \sqrt{9-x}) = 2x-12 \]
\[ 22-2x = 2(\sqrt{x+1} \cdot \sqrt{9-x}) \]
\[ 11-x = \sqrt{(x+1)(9-x)} \]
Itt megint kikötést kell tenni.
\[ -x + 11 \geq 0 \]
\[ x \leq 11 \]
Bár ez nem változtat az előző kikötésen.
\[ x^2 + 121 - 22x = -x^2 + 8x + 9 \]
\[ 2x^2 -30x +112 = 0 \]
\[ x^2 -15x +56 = 0 \]
\[ (x-7)(x-8) = 8 \]
$ x_1 = 7 $ és $ x_2=8 $ ezek $\in [6,9]$ tehát mindkettő megoldás.


\myparagraph{6. (k)}
\[ \sqrt{x^2+4x} > 2-x \]
Kikötés: $ x^2 +4x \geq 0 \Rightarrow x \leq -4$ vagy $x \geq 0$
\begin{enumerate}
  \item eset:
    \[ 2-x \leq 0 \]
    $x \geq 2 \Rightarrow x \in [2, \infty]$ ez megoldás.
  \item eset:
    \[ 2-x > 0 \]
    \[ x < 2 \]
\end{enumerate}
\[ \sqrt{x^2+4x} > 2-x \]
\[ x^2 + 4x > x^2 - 4x +4 \]
\[ 8x > 4 \]
\[ x > \dfrac{1}{2} \]
Így $ \Rightarrow x \in \left(\dfrac{1}{2}, 2\right)$, ez az intervallum még jó
lesz de más már nem.


\section{Logaritmikus, exponenciális egyenletek}
\setcounter{subsection}{1}
\subsection{Feladatok}
\subsubsection{Órai feladatok}


\myparagraph{2.}
\[ 2^{x+3} + 4^{1-\nicefrac{x}{2}} = 33 \]
Mindent $2^x$ tagra kell vinni.
\[ 2^{x+3} = 2^3 \cdot 2^{\nicefrac{x}{2}} = 8 \cdot 2^x \]
\[ 4^{1-x} = 4 \cdot 4^{-\frac{x}{2}} = 4 \cdot \dfrac{1}{4^{\frac{x}{2}}} =
  4 \cdot \dfrac{1}{2^x} = 4 \cdot (2^2)^{\frac{2x}{2}} = 4 \cdot 2^x
    \Rightarrow 8 \cdot 2^x + 4 \cdot \dfrac{1}{2^x} = 33
\]
\[ 2^x := y \Rightarrow \]
\[ 8y + 4 \cdot \dfrac{1}{y} = 33 \]
\[ 8y^2 - 33y + 4 = 0 \]
\[ 8(y-\dfrac{1}{8})(y-4) =0 \]
\[ y_1 = 4 \text{ és } y_2 = \dfrac{1}{8} \]
ha $y_1 = 4$ akkor $x_1 = 2$ \\
ha $y_2 = \dfrac{1}{8}$ akkor $x_2 = -3$ \\
Ez a két megoldás lesz.


\myparagraph{3. (c)}
\[ 3^{x+2} \cdot 2^x - 2 \cdot 36^x + 18 = 0 \]
\[ 9 \cdot 3^x \cdot 2^x = 9 \cdot 6^x \]
\[ 2 \cdot (6^2)^x= 2 \cdot 6^{2x} \]
\[ 9 \cdot 6^x - 2 \cdot 6^{2x} + 18 = 0 \]
\[ 6x := y \Rightarrow 9y - 2y^2 + 18 = 0 \]
\[ 2y^2 - 9y - 18 = 0 \]
\[ 2(y+\dfrac{3}{2})(y-6) = 0 \]
\[ y_1=-\dfrac{3}{2} \text{ és } y_2 = 6 \]
$ 6^x = -\dfrac{3}{2}$ ilyen nem létezik az $\mathbb{R}$-ben ezért $ x_2 = 1 $
  ez az egy megoldás van.


\myparagraph{3. (f)}
\[ 4^{x+1} - 9 \cdot 2^x + 2 > 0 \]
\[ 4 \cdot (2^x)^2 - 9 \cdot 2^x + 2 > 0 \quad 2^x:=y\]
\[ 4y^2 - 9y + 2 > 0 \]
\[ y_1 = 2 \quad y_2 = \dfrac{1}{4} \]
1. eset: $y > 2 \rightarrow x > 1$ \\
2. eset: $y < \dfrac{1}{4} \Rightarrow x < -2 $


\myparagraph{8.}
\[ 3^{2+\log_9 25} + 25^{1-\log_5 2} + 10^{-\lg 4} = 3^{\log_9 81 + \log_9 25} +
  25^{\log_5 5 - \log_5 2} + 10^{\lg 4^{-1}} = \]
\[ \left(9^{\frac{1}{2}}\right)^{\log_9 (81 \cdot 25)} + \left(5^2\right)^{\log_5 \frac{5}{2}} +
  \dfrac{1}{4} = (81 \cdot 25)^{\frac{1}{2}} + \left(\dfrac{5}{2}\right)^2 + \dfrac{1}{4} \]

\myparagraph{15. (c)}
\[ \log_3 (x+1) - \log_3 (x+10) = 2 \cdot \log_3 \frac{9}{2} - 4 \]
\[ \log_3 \dfrac{(x+1)}{(x+10)} = \log_3 \left(\dfrac{9}{2}\right)^2 - \log_3 81 \]
\[ \log_3 \dfrac{(x+1)}{(x+10)} = \log_3 \dfrac{\left(\dfrac{81}{4}\right)}{81} \]
Mivel a logaritmus szigorúan monoton függvény. $\Longrightarrow$
\[ \dfrac{x+1}{x+10} = \dfrac{1}{4} \]
\[ 4x + 4 = x + 10 \]
\[ 3x = 6 \]
\[ x = 2 \]

\myparagraph{15. (e)}
\[ \log_{32} (2x) - \log_8 (4x) + \log_2 (x) = 3 \]
\[ \log_{32} (2) + \log_{32} (x) - (\log_8 (4) + \log_8 (x)) + \log_2 (x) = 3 \]
\[ \dfrac{1}{5} + \dfrac{\log_2 (x)}{\log_2 (32)} - \dfrac{2}{3} + \dfrac{\log_2 (x)}{\log_2 (8)} +
  \log_2 (x) = 3\]
\[ \dfrac{1}{5} + \dfrac{\log_2 (x)}{5} - \dfrac{2}{3} + \dfrac{\log_2 (x)}{3} + \log_2 (x) = 3 \]
\[ \log_2 (x) \cdot \dfrac{13}{15} - \dfrac{7}{15} = 3 \]
\[ \log_2 (x) = \dfrac{52}{15} \cdot \dfrac{15}{13} \]
\[ \log_2 (x) = 4 \]
\[ x = 2 \]

\clearpage
\myparagraph{15. (j)}

\begin{wrapfigure}[0]{R}{0pt}
  \begin{tikzpicture}[scale=1.2]
    \draw [step=0.5, help lines] (-1, -3) grid (3, 3);
    \draw [->, thick] (-1, 0) -- (3, 0);
    \draw [->, thick] (0, -3) -- (0, 3);
    \draw [thick] (1, -0.15) -- (1, 0.15);
    \node [] at (1, 0.3) {\small{1}};
    \draw [green, very thick, domain=0.124:3,samples=50] plot (\x, {log2(\x)});
    \draw [red, very thick, domain=3:0.124,samples=50] plot (\x, {ln(\x)/ln(0.5)});
    \node [above right] at (0.5, 1) {$\log_{\frac{1}{2}} x$};
    \node [below right] at (0.5, -1) {$\log_{2} x$};
  \end{tikzpicture}
\end{wrapfigure}
\[ \log_{\frac{1}{2}} \left( \dfrac{3-x}{3x-1} \right) \geq 0 \]
\[ \log_{\frac{1}{2}} \left( \dfrac{3-x}{3x-1} \right) \geq \log_{\frac{1}{2}} 1 \]
\textbf{Kikötés:} $ 3x -1 \neq 0 \Rightarrow x \neq \dfrac{1}{3} $ \\[2em]
A $\log_{\frac{1}{2}} x$ is az 1-nél megy át az $x$ tengelyen de \\
tükörképe lesz a $\log_2 x$-nek. Mivel ez egy szigorúan \\
monoton csökkenő függvény, megfordul az \\
egyenlőtlenségnél a reláció.

\[ \dfrac{3-x}{3x-1} \leq 1 \]
Egy tört akkor pozitív ha a számláló és a nevező is vagy pozitív vagy negatív.
%TODO: % A 11.oldalon ez van : Tört  <= 1
% te meg úgy oldod meg innentől, mint ha Tört >=0 lenne a kérdés
\begin{enumerate}[i.]
  \item Mindkettő pozitív
  \[
  \begin{rcases*}
    3-x \geq 0 \\
    3x -1 \geq 0 \\
  \end{rcases*} \Rightarrow x \in \left[ \frac{1}{3}, 3 \right]
  \]
  \item Mindkettő negatív
  \[ 3-x \leq 0 \Rightarrow x \geq 3 \]
  \[ 3x -1 \leq 0 \Rightarrow x \leq \frac{1}{3} \]
  Ilyen az $\mathbb{R}$-ben nincs tehát marad az előző kikötés.
  $x \in \left( \frac{1}{3}, 3 \right]$ (Mert az első kikötés szerint
  $ x \neq \frac{1}{3}$).
\end{enumerate}
Ezt még hozzá kell tenni a kikötéshez.
\[ \dfrac{3-x}{3x-1} \leq 1 \]
\[ 3-x \leq 3x-1 \]
\[ -4x \leq -4 \]
\[ x \geq 1 \]
Tehát a megoldás: $x \in \left[ 1, 3 \right]$.

\clearpage
\section{Trigonometrikus azonosságok, egyenletek, \\ egyenlőtlenségek}
\setcounter{subsection}{1}
\subsection{Feladatok}
\subsubsection{Órai feladatok}


\myparagraph{4. (a)}
\[ \sin(4x) = \sin(x) \]
\begin{enumerate}
  \item megoldási lehetőség:
  \[ 4x = x \]
  \[ 4x = x + 2k\pi \quad k \in \mathbb{Z} \]
  \[ x = \dfrac{2k\pi}{3} \quad k \in \mathbb{Z} \]
  Így ezek jó megoldások: $k = 0$ esetén $x = 0$, $k = 1$ esetén $x =
    \dfrac{2\pi}{3}$ ... \\
  Ezek a triviális megoldás.
  \item megoldási lehetőség:
  \[ 4x = \pi - x + 2k\pi \quad k \in \mathbb{Z} \]
  \[ x = \dfrac{\pi + 2k\pi}{5} \]
  Így $k = 0$ esetén $x = \dfrac{\pi}{5}$, $k = 1$ esetén $x = \dfrac{3\pi}{5}$
\end{enumerate}


\myparagraph{4. (d)}
\[ \cos(2x) - 3 \cdot \cos(x) + 2 = 0 \]
\[ \cos^2(x) - \sin^2(x) - 3 \cdot \cos(x) + 2 = 0 \]
\[ \cos^2(x) - (1- \cos^2(x)) 3 \cdot \cos(x) + 2 = 0 \]
\[ 2 \cdot \cos^2(x) - 3 \cdot \cos(x) + 1 = 0 \]
Ez pedig már egy sima másodfokú egyenlet $\cos(x)$-re.
\[ \cos(x) := a \Rightarrow 2a^2 - 3a + 1 = 0 \Rightarrow a_1 = 1, a_2 = \frac{1}{2} \]
\[ \cos(x) = 1 \text{ vagy } \cos(x) = \frac{1}{2}\]
Mindkettő lehetséges:
\[ \cos(x) = 1 \Rightarrow x = 2k\pi \qquad k \in \mathbb{Z} \]
\[ \cos(x) = \dfrac{1}{2} \Rightarrow x = \dfrac{\pi}{3} + 2k\pi \qquad k \in \mathbb{Z} \]
\textbf{Viszont!} Mivel $\cos(\beta) = \cos(-\beta)$, ezért
\[ \cos(x) = \dfrac{1}{2} \Rightarrow x = -\dfrac{\pi}{3} + 2k\pi \qquad k \in \mathbb{Z} \]

\myparagraph{4. (h)}
\[ \sqrt{3} \cdot \sin(x) + \cos(x) = \sqrt{3} \]
\[ \sqrt{3} \cdot \sin(x) + 1 \cdot \cos(x) = \sqrt{3} \quad /:2 \]
\[ \dfrac{\sqrt{3}}{2} \cdot \sin(x) + \dfrac{1}{2} \cdot \cos(x) = \dfrac{\sqrt{3}}{2} \]
\[ \sin\left(\dfrac{\pi}{3}\right) \cdot \sin(x) +
\cos\left(\dfrac{\pi}{3}\right) \cdot \cos(x) = \dfrac{\sqrt{3}}{2} \]
\textbf{Megjegyzés:} $\cos(\alpha - \beta) =
  \sin(\alpha) \cdot \sin(\beta) + \cos(\alpha) \cdot \cos(\beta) $
\[ \cos\left( x - \dfrac{\pi}{3} \right) = \dfrac{\sqrt{3}}{2} \]
Melyek azok a szögek amelyeknek a koszinusza $\dfrac{\sqrt{3}}{2}$ ?

%\centering % TODO Kijavítani a középre igazítást
\begin{tikzpicture}[scale=1.5]
  \begin{axis}%
    [grid=both,
     minor tick num=4,
     grid style={line width=.1pt, draw=gray!10},
     major grid style={line width=.2pt,draw=gray!50},
     axis lines=middle,
     enlargelimits={abs=0.2}
    ]
    \addplot[domain=-1:3,samples=50,smooth,red] {cos(deg(pi*x))};
    \draw [dotted] (-1, 0.866025) -- (3, 0.866025);
    \node [above] at (-0.7, 0.866025) {\tiny $\frac{\sqrt{3}}{2}$};
    \node [below] at (0.5, -0.05) {\tiny $\frac{\pi}{2}$};
  \end{axis}
\end{tikzpicture}
%\par

\clearpage

\begin{enumerate}[i.]
  \item
  \[ x - \dfrac{\pi}{3} = \dfrac{\pi}{6} + 2k\pi \qquad k \in \mathbb{Z} \]
  \[ x = \dfrac{\pi}{2} + 2k\pi \]
  \item
  \[ x - \dfrac{\pi}{3} = -\dfrac{\pi}{6} + 2k\pi \qquad k \in \mathbb{Z} \]
  \[ x = \dfrac{\pi}{6} + 2k\pi \]
\end{enumerate}

\myparagraph{7. (c)}
\[ \dfrac{2 \cdot \sin(x) + 1}{2 \cdot \cos(x)} \leq 0 \]
Kikötés: $ x \neq \dfrac{\pi}{2} + 2k\pi \qquad k \in \mathbb{Z} $
\begin{enumerate}
  \item eset
  \[ 2 \cdot \sin(x) + 1 \leq 0 \Rightarrow \sin(x) \leq -\dfrac{1}{2}\]
  \[ cos(x) > 0 \]
  \[ x \in \left\{ \left[0, \dfrac{\pi}{2}\right)
    \cup \left(\dfrac{3\pi}{2}, 2\pi\right] \right\} + 2k\pi \qquad k \in \mathbb{Z} \]
  Itt lesz a $\cos(x) > 0$.
  Számoljuk ki hol lesz $\sin(x) = -\dfrac{1}{2}$ vagy $\sin(x) = \dfrac{1}{2}$
  \[ x \in \left[\dfrac{7\pi}{6}, \dfrac{11\pi}{6}\right] \]
  A kettő metszetét kell venni:
  \[ x \in \left(\dfrac{3\pi}{2}, \dfrac{11\pi}{6}\right] + 2k\pi \qquad k \in \mathbb{Z} \]
  \item eset
  Itt megfordulnak a relációk. HF.
\end{enumerate}


\clearpage
\section{Nagyságrend-őrző becslések és\\ függvények további becslései}
\setcounter{subsection}{1}
\subsection{Feladatok}
\subsubsection{Órai feladatok}

\myparagraph{1. (a)}
\[ 4x^5 - 3x^4 - 2x^2 - 5 \]
\begin{enumerate}[i.]
  \item Felső becslés
  Mivel a negatív tagok csak csökkentik az értéket:
  \[ 4x^5 - 3x^4 - 2x^2 - 5 \leq 4x^5 \qquad x > 0 \text{ esetén} \]
  \[ \Rightarrow M = 4,\; n = 5,\; R = 0 \]
  \item Alsó becslés
  \[ 4x^5 - 3x^4 - 2x^2 - 5 \geq 6x^5 \qquad x > 0 \text{ esetén} \]
  \[ \Rightarrow m = 6,\; n = 5,\; R = 0 \]
\end{enumerate}

\myparagraph{1. (b)}
\[ 2x^3 - 3x^2 + 6x + 7 \]
\begin{enumerate}[i.]
  \item Felső becslés
  Első lépésként elhagyjuk azokat amelyek csökkentik:
  \[ 2x^3 - 3x^2 + 6x + 7 \leq 2x^3 + 6x + 7 \qquad x > 0 \text{ esetén} \]
  \[ \leq 2x^3+6x^3+7x^3 = 15x^3 \qquad x \leq 1 \Rightarrow M = 15,\; n = 3,\; R = 1 \]
  \item Alsó becslés
  \[ 2x^3 - 3x^2 + 6x + 7 \geq 2x^3 - 3x^2 \qquad x > 0 \text{ esetén} \]
  \[ 2x^3 - 3x^2 \geq 2x^3 - x^3 = x^3 \qquad x > 3 \text{ esetén} \]
  \[ \Rightarrow m = 1,\; n = 3,\; R = 3 \]
  Másik megoldási lehetőség:
  \[ 2x^3 - 3x^2 + 6x + 7 \geq 2x^3 - (3x^2 - 6x - 7) \]
  \[ 3x^2 - 6x - 7 \Rightarrow \dfrac{6 \pm 2\sqrt{30}}{6} \geq 3\]
  \[ 2x^3 - (3x^2 - 6x - 7) \geq 2x^3 \Rightarrow m = 2,\; n = 3,\; R = 3 \]
\end{enumerate}

\myparagraph{1. (c)}
\[ 6x^5 + 7x^4 + 10x^3 + x^2 + 2x + 3 \]
\begin{enumerate}[i.]
  \item Felső becslés
  \[ 6x^5 + 7x^4 + 10x^3 + x^2 + 2x + 3 \leq 29x^5 \qquad x \geq 1 \text{ esetén}\]
  \[ \Rightarrow M = 29,\; n = 5,\; R = 1 \]
  \item Alsó becslés
  \[ 6x^5 + 7x^4 + 10x^3 + x^2 + 2x + 3 \geq 6x^5 \qquad x \geq 0 \text{ esetén}\]
  \[ \Rightarrow m = 6,\; n = 5,\; R = 0 \]
\end{enumerate}

\myparagraph{3. (a)}
\[ R(x) = \dfrac{3x^4 + 2x^3 + 5x^2 + 7x + 7}{5x^2 - 3x -10} \]
A nevezőt növeljük, a számlálót csökkentjük.
\[ \geq \dfrac{m \cdot x^4}{M \cdot x^2} = \dfrac{m}{M} \cdot x^2\]
\[ 3x^4 + 2x^3 + 5x^2 + 7x + 7 \geq 3x^4 \qquad x \geq 0 \text{ esetén} \]
\[ 5x^2 - 3x -10 \leq 5x^2 \qquad x \geq 0 \text{ esetén} \]
\[ R(x) \geq \dfrac{3x^4}{5x^2} = \dfrac{3}{5}x^2 \qquad x \geq 0 \text{ esetén} \]
Másik lehetőség:
\[ 3x^4 + 2x^3 + 5x^2 + 7x + 7 \leq 23x^4 \qquad x > 1 \text{ esetén} \]
\[ 5x^2 - 3x -10 = 5x^2 -(3x + 10) \geq 5x^2-x^2 = 4x^2 \qquad x \geq 5 \text{ esetén} \]
Így:
\[ R(x) \leq \dfrac{23x^4}{4x^2} = \dfrac{23x^2}{4} \qquad x \geq 5 \text{ esetén} \]
%TODO: Ellenőrzés


\clearpage
\section{Kijelentések, kvantorok, logikai állítások I.}
\setcounter{subsection}{1}
\subsection{Feladatok}
\subsubsection{Órai feladatok}


\myparagraph{1. (h)}
\[ (\neg A \lor B) \Rightarrow B = A \lor B \]
\begin{tabular}{c|c||c|c|c||c}
  $A$ & $B$ & $\neg A$ & $(\neg A \lor B)$ & $(\neg A \lor B) \Rightarrow B$ & $A \lor B$ \\
    \hline
  I & I & H & I & I & I \\ \hline
  H & H & I & I & H & H \\ \hline
  I & H & H & H & I & I \\ \hline
  H & I & I & I & I & I \\
\end{tabular}

\myparagraph{3. (a)}
\[ x = 0 \text{ és } y = 0 \Rightarrow x^2 + y^2 = 0 \qquad \text{Ez igaz.} \]
Megfordítva:
\[ x = 0 \text{ és } y = 0 \Leftarrow x^2 + y^2 = 0 \qquad \text{Ez is igaz.} \]

\myparagraph{3. (b)}
\[ x \cdot y = x \cdot z \Rightarrow y = z \qquad \text{Ez hamis.} \]
Mert ha pl: $x = 0$. Megfordítva:
\[ x \cdot y = x \cdot z \Leftarrow y = z \qquad \text{Így már viszont igaz.} \]

\myparagraph{3. (c)}
\[ x > y^2 \Rightarrow x > 0 \qquad \text{Ez igaz.} \]
Megfordítva:
\[ x > y^2 \Leftarrow x > 0 \qquad \text{Hamis.} \]

\myparagraph{5. (a)}
\[ \forall n : \frac{1}{n} < 0.01 \qquad \text{Hamis.} \]
Tagadjuk:
\[
  \neg(\forall n : \frac{1}{n} < 0.01)
  \rightarrow \exists n: \neg(\frac{1}{n} < 0.01)
  \rightarrow \exists n : \frac{1}{n} \geq 0.01
\]

\myparagraph{5. (b)}
\[ \exists n : \frac{1}{n} < 0.01 \qquad \text{Igaz.} \]
Tagadjuk:
\[ \neg (\exists n : \frac{1}{n} < 0.01) \rightarrow
  \forall n : \frac{1}{n} \geq 0.01 \qquad \text{Hamis.} \]

\myparagraph{7. (b)}
\[ \dfrac{2n^3+3}{n^5-3n^4-7n^3+2n^2-10n+1} < 0.05 \]
\[
  \exists N : \forall n \geq N : \dfrac{2n^3+3}{n^5-3n^4-7n^3+2n^2-10n+1} < 0.05
\]
Felső és alsó becslés alapján eldönthető hogy igaz-e.
\[ n \geq 1 \text{ esetén} \qquad \frac{n^3}{\frac{1}{2}n^5} \leq \frac{10}{n^2} < 0.05\]


\clearpage
\section{Kijelentések, kvantorok, logikai állítások II.}
\setcounter{subsection}{1}
\subsection{Feladatok}
\subsubsection{Órai feladatok}

\myparagraph{1. (a)}
\[ a + b = 0 \iff a^2 + b^2 = -2ab \]
Megnézzük mindkét irányt. \\
$\Rightarrow$: Igaz. Ugyanis: $a^2 + b^2 = -2ab \iff (a+b^2 = 0)$ \\
$\Leftarrow$: Ez is igaz. \\
Tehát az ekvivalencia is igaz.

\myparagraph{1. (b)}
\[ a + b = 1 \iff a^2 + b^2 = 1 - 2ab \]
\[ a^2 + b^2 = 1 - 2ab \]
\[ a^2 + b^2 + 2ab = 1 \]
\[ (a+b)^2 = 1 \]
$\Rightarrow$: Igaz. \\
$\Leftarrow$: Hamis. Ugyanis ha $a+b=-1$ akkor is a négyzetre emelés. \\
Tehát az ekvivalencia hamis.

\myparagraph{1. (c)}
\[ x = -1 \iff x^2 + x = 0 \]
$\Rightarrow$: ez az irány igaz. \\
$\Leftarrow$: Hamis, mert $x$ lehetne $0$ is. \\
Tehát az ekvivalencia hamis.

\myparagraph{1. (i)}
\[ |x-5| < 2 \iff 3 < x < 7 \]
$\Rightarrow$: ez az irány igaz. Ugyanis $|x-5| < 2 \iff -2 < x-5 < 2
  \iff 3 < x < 7 $ \\
$\Leftarrow$: ez is igaz. \\
Tehát az ekvivalencia igaz.

\myparagraph{2. (a)}
\[ \forall x \in \mathbb{R}: \quad (x-1)^2+(x-5)^2+(x-12)^2 \geq 62 \]
A bal oldalt valahogy át kéne alakítani. Bontsuk fel a zárójeleket.
\[ x^2-2x+1+x^2-10x+25+x^2-24x+144 \geq 62 \]
\[ 3x^2 - 36x + 170 \geq 62 \qquad /-62 \]
\[ 3x^2 - 36x + 108 \geq 0  \qquad /:3  \]
\[  x^2 - 12x + 36  \geq 0 \]
\[ (x-6)^2 \geq 0 \qquad \text{(\textbf{Megjegyzés:} mindenhol ekvivalens
  átalakításokat végeztünk)}\]
Mivel egy négyzetszám mindig nagyobb mint nulla ezért az állítás igaz.

\myparagraph{3. (a)}
\[ f(x) = \big|1-|x|\big| \qquad x \in [-3, 2) \]
\[ \forall x \in \mathcal{D}_f: f(x) \geq 0 \]
Ez igaz, mivel abszolút értek.

\myparagraph{3. (b)}
\[ \forall x \in \mathcal{D}_f: f(x) \leq 2 \]
Becsüljük felülről.
\[ f(x) \leq 1-|x| \leq 0 \text{ vagy } 1 - |x| \geq 0 \]
\[ \big|1-|x|\big| \leq 2 \;\text{ ha }\; -3 \leq x < 1 \text{ vagy } 1 < x < 2 \]
Egyszerűen látható a megoldás ha ábrázoljuk: \\[1.2em]
\begin{tikzpicture}[scale=1.8]
  \draw [step=0.5, help lines] (-4, -1) grid (3, 3);
  \draw [->, thick] (-4, 0) -- (3, 0);
  \draw [->, thick] (0, -1) -- (0, 3);

  \draw [green, very thick, domain=-4:3, samples=8] plot (\x, {abs(1-abs(\x))});

  \draw [thick] (-0.1, 1) -- (0.1, 1);
  \draw [thick] (-0.1, 2) -- (0.1, 2);
  \node [right] at (0.1, 1) {$1$};
  \node [right] at (0.1, 2) {$2$};

  \draw [thick] (1, 0.1) -- (1, -0.1);
  \draw [thick] (2, 0.1) -- (2, -0.1);
  \node [below] at (1, -0.1) {$1$};
  \node [below] at (2, -0.1) {$2$};

  \draw [thick] (-1, 0.1) -- (-1, -0.1);
  \draw [thick] (-2, 0.1) -- (-2, -0.1);
  \draw [thick] (-2, 0.1) -- (-2, -0.1);
  \node [below] at (-1, -0.1) {$-1$};
  \node [below] at (-2, -0.1) {$-2$};
  \node [below] at (-3, -0.1) {$-3$};

  \draw [very thick] (2, 0) -- (2,1);
  \draw [fill=white] (2, 1) circle (0.5mm);

  \draw [very thick] (-3, 0) -- (-3,2);
  \draw [fill=black] (-3, 2) circle (0.5mm);
\end{tikzpicture}
\\ Tehát igaz.


\myparagraph{3. (c)}
\[ !\exists a \in \mathcal{D}_f: \forall x \in \mathcal{D}_f \quad f(a) \leq f(x) \]
Ez egyszerűbben megfogalmazva: pontosan egy minimuma (Dimat1) van. Az előző ábrából
könnyen leolvasható hogy ez hamis. Ugyanis az $a=1$ és $a=-1$ is minimum.


\myparagraph{3. (d)}
\[ \exists a \in \mathcal{D}_f: \forall x \in \mathcal{D}_f \quad f(a) \leq f(x) \]
Mivel itt nincs kikötve az egyértelműség ez már igaz.


\myparagraph{3. (e)}
\[ !\exists a \in \mathcal{D}_f: \forall x \in \mathcal{D}_f \quad f(a) \geq f(x) \]
Ez a 3/c. ellentéte, vagyis itt maximumot keresünk. Van ilyen az $a=-3$ amiből
egy van, így igaz az állítás.


\myparagraph{3. (f)}
\[ !\exists b \in \mathcal{D}_f: f(b) = 1 \]
Ez hamis mert három helyen is egy a függvény értéke.

\myparagraph{3. (g)}
\[ \exists b \in \mathcal{D}_f: f(b) = 0 \]
Ez igaz mert két helyen nulla a függvény értéke.


\myparagraph{3. (h)}
\[ !\exists x \in \mathcal{D}_f: f(x) = x \]
Az $f(x) = x$ függvény az origón 45°-ban áthaladó egyenes. Ez pedig egy helyen
metszi az eredeti függvényt tehát az állítás igaz.


\myparagraph{3. (i)}
$\forall c \in \mathbb{R}$ esetén az $f(x) = c$ egyenletnek van legalább egy
megoldása. \\
Ez hamis ugyanis pl: $f(x)=125$-nek nincs megoldása a -3 és 2 intervallumban.

\myparagraph{3. (j)}
Az $f(x) = c$ egyenletnek van legalább egy megoldása $\iff c \in [0,2]$. \\
$\Rightarrow$: ebbe az irányba igaz, ugyanis ha $c \notin [0,2]$ nem lesz
megoldása $f(x)$-nek. \\
$\Leftarrow$: ez is igaz.


\myparagraph{3. (l)}
Az $f(x) = c \; (c \in \mathbb{R})$ egyenletnek akkor és csak akkor van pontosan 3
darab megoldása, ha $c = 1$. \\
$\Leftarrow$: hamis mert nincs a függvénynek sehol három megoldása. \\
$\Rightarrow$: ez viszont igaz mert ha egy implikáció jobb oldala hamis az csak
úgy lehet ha a bal oldal igaz. \\[1.3em]
\begin{tabular}{c|c|c}
  $P$ & $Q$ & $P \Rightarrow Q$ \\ \hline
  I & I & I \\ \hline
  I & H & H \\ \hline
  H & I & I \\ \hline
  H & H & I
\end{tabular}


\clearpage
\tableofcontents
\end{document}

