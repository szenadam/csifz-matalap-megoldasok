\documentclass[12pt,a4paper,fleqn]{article}
\usepackage[a4paper,left=3cm,right=2cm,top=2.5cm,bottom=2.5cm]{geometry}
\usepackage{amsfonts}
\usepackage{mathtools}
\usepackage{fancyhdr}
\setlength{\mathindent}{0pt}

\author{Et Alii}
\pagestyle{fancy}
\fancyhf{}
\lhead{\leftmark}
\rhead{\thepage}

\title{Matematikai Alapok \\ \textbf{Megoldások}}

\begin{document}
\maketitle
\newpage

\section{Algebrai és Gyökös kifejezések I.}
\setcounter{subsection}{1}
\subsection{Feladatok}
\subsubsection{Órai feladatok}
\textbf{3. (b)}
\begin{flalign*}
  & \frac{a}{a^3+a^2b+ab^2+b^3} + \frac{b}{a^3-a^2b+ab^2-b^3} + \frac{1}{a^2-b^2} - \frac{1}{a^2+b^2} - \frac{a^2+3b^2}{a^4-b^4} = 0  \\\\
  & \frac{a}{a^2(a+b)+b^2(a+b)} + \frac{b}{a^2(a-b)+b^2(a-b)} + \frac{1}{(a+b)(a-b)} - \frac{1}{a^2+b^2} - \frac{a^2+3b^2}{(a^2-b^2)(a^2+b^2)} = 0 \\\\
  & \frac{a}{(a+b)(a^2+b^2)} + \frac{b}{(a-b)(a^2+b^2)} + \frac{1}{(a+b)(a-b)} - \frac{1}{a^2+b^2} - \frac{a^2+3b^2}{(a+b)(a-b)(a^2+b^2)} = 0 \\\\
  & \frac{a(a-b)+b(a+b)+(a^2+b^2)-(a+b)(a-b)-(a^2+3b^2)}{(a+b)(a-b)(a^2+b^2)} = 0 \\\\
  & \frac{a^2-ab+ab+b^2+a^2+b^2-a^2+b^2-a^2-3b^2}{(a+b)(a-b)(a^2+b^2)} = 0
\end{flalign*} \\
\textbf{4. (a)}
\begin{enumerate}
  \item Első lehetőség: $a+b+c=0$ kiemelhető így $(a+b+c) \cdot (a^2+b^2-ab)$
  \item Második lehetőség: $(a+b+c)$-ből kifejezzük az egyiket $\rightarrow c = -a - b$ és ezt kell behelyettesíteni. \\
  $a^3 + a^2(-a-b) -ab(-a-b) + b^2(-a-b) + b^3$ \\
  $ = a^3 - a^3 - a^2b + a^2b +ab^2 - ab^2 - b^3 + b^3$ \\
  $ = 0$ \\
\end{enumerate}
\textbf{6.} \\\\
$$ \dfrac{x^3-x-y^3+y+xy^2-x^2y}{x^3+x-y^3-y+xy^2-x^2y} =
  \dfrac{xy(y-x)+(y-x)+(x-y)(x^2+xy+y^2)}{xy(y-x)-(y-x)+(x-y)(x^2+xy+y^2)} $$
  $$= \dfrac{(y-x)(xy+1-x^2-y^2-xy)}{(y-x)(xy-1-x^2-y^2-xy)}
  = \dfrac{1-x^2-y^2}{-1-x^2-y^2} = \dfrac{-1(-1+x^2+y^2)}{-1(1+x^2+y^2)} = \dfrac{x^2+y^2-1}{x^2+y^2+1}$$
Ha $x = \dfrac{k(1-z^2)}{(1+z^2)}$ akkor $y = \dfrac{2k \cdot z}{1+z^2}$.
Bizonyítsuk be hogy behelyettesítés után a kifejezés nem függ a $z$ értékétől (vagyis ki fog esni)!
$$ \dfrac{x^2+y^2-1}{x^2+y^2+1} = \dfrac{x^2+y^2+1-2}{x^2+y^2+1} = 1 - \dfrac{2}{x^2+y2+1} $$
Behelyettesítve:
$$ x^2+y=\dfrac{(k(1-z^2))^2}{(1+z^2)^2} + \dfrac{(2kz)^2}{(1+z^2)^2} = \dfrac{k^2(1-z^2)^2+4k^2z^2}{(1+z^2)^2}$$
$$ = \dfrac{k^2(1+z^4-2z^2+4z^2)}{(1+z^2)^2} = \dfrac{k^2(z^2+1)^2}{(1+z^2)^2} = k^2$$
Így $1-\dfrac{2}{k^2+1}$ ami valóban független a $z$-től. \\\\
\textbf{12. (c)} \\\\
$$ \left(\sqrt{x} - \dfrac{\sqrt{xy}+y}{\sqrt{x}+\sqrt{y}}\right) \cdot \left( \dfrac{\sqrt{x}}{\sqrt{x}+\sqrt{y}} + \dfrac{\sqrt{y}}{\sqrt{x}-\sqrt{y}} + \dfrac{2\sqrt{xy}}{x-y} \right) $$
$$ = \left(\dfrac{\sqrt{x}(\sqrt{x}+\sqrt{y})-(\sqrt{xy}+y)}{\sqrt{x} + \sqrt{y}} \right) \cdot \left( \dfrac{\sqrt{x}(\sqrt{x}-\sqrt{y})+\sqrt{y}(\sqrt{x}+\sqrt{y})+2\sqrt{xy}}{(\sqrt{x}-\sqrt{y})(\sqrt{x}+\sqrt{y})} \right)$$
$$ = \left( \dfrac{x-y}{\sqrt{x}+\sqrt{y}} \right) \cdot \left( \dfrac{x-\sqrt{xy}+\sqrt{xy}+y+2\sqrt{xy}}{(\sqrt{x}-\sqrt{y})(\sqrt{x}+\sqrt{y})} \right)$$
$$ = \left( \dfrac{x-y}{\sqrt{x}+\sqrt{x}} \cdot \left( \dfrac{x+y+2\sqrt{xy}}{(\sqrt{x}-\sqrt{y})(\sqrt{x}+\sqrt{y})} \right) \right) = \left( \dfrac{x-y}{\sqrt{x}+\sqrt{y}} \cdot \left( \dfrac{(\sqrt{x}+\sqrt{y})^2}{(\sqrt{x}+\sqrt{y})(\sqrt{x}-\sqrt{y})} \right) \right)$$
$$ = \dfrac{x-y}{\sqrt{x} - \sqrt{y}} = \sqrt{x} + \sqrt{y}$$ (mert: $x-y=(\sqrt{x}-\sqrt{y}(\sqrt{x}+\sqrt{x}))$)


\end{document}


